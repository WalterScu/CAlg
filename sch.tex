% !TEX root=./CAlg.tex
\section{Topological properties of prime spectrum}
Let $R$ be a ring.
\[
\spec R = \big\{\text{all prime ideals of }R  \big\}
\]
We can give $\spec R$ topology called Zariski topology, which is the origin of modern algebraic geometry. Closed sets of Zariski topology is defined with form 
\[
V(I) = \big\{ \mathfrak{p} \in \spec R | I \subseteq \mathfrak{p}\big\}
\]
where $I$ is arbitrary ideal of $R$.

It is homework to check that these subsets of $\spec R$ satisfy closed set axioms
\begin{enumerate}
\item $V(I_1) \cup V(I_2 )= V(I_1 \cap I_2)$ 
\item $\bigcap V(I) = V(\sum I)$
\item $\emptyset = V(1)$ and $\spec R = V(0)$
\end{enumerate}

\begin{ex}
$\spec \mathbb{Z}$

$\spec k[ x]$, where $k$ is a algebraic closed field.
\end{ex}
\[ D(I) = \spec R \setminus V(I)\]
are used to denote the open sets of $\spec R$ in Zariski topology. In particular, if $I =(f)$ is principal ideal generated by $x$, then $D(I)= D(f)$ is called a principal open set.
\begin{secprop}
$V(I) = V(\sqrt{I}); D(I)= D(\sqrt{I})$ 
\end{secprop}
\begin{proof}
If $I$ is contained in a prime ideal $\mathfrak{p}$, then $ \forall x \in \sqrt{I}, \exists n >0, x^n \in I$. So $x^n \in \mathfrak{p}$. Since $\mathfrak{p}$ is prime, $x \cdot x^{n-1} \in \mathfrak{p}$ implies that $x \in \mathfrak{p}$ or $x^{n-1} \in \mathfrak{p}$, then by induction of $n$, we can conclude that $x \in \mathfrak{p}$. Hence $\sqrt{I} \subseteq \mathfrak{p}$. Hence $V(I) = V( \sqrt{I})$.
\end{proof}
\begin{secprop}
Principal open subsets of $\spec R$ form topology base of Zariski topology.
\end{secprop}
\begin{proof}
\begin{enumerate}
\item $\bigcup_{f \in I} = D(I)$ for any open subset $D(I)$ of  $\spec R$. 
\item For any ideal $J$, there is a principal ideal $(f)$ such that $D_f \subseteq D(I)$. For example, we can choose any element $f$ of $I$.
\end{enumerate}
\end{proof}

\begin{secprop}
$\spec R$ is quasi-compact.
\end{secprop}
\begin{proof}
	It is sufficient to prove for any principal open cover $\bigcup_i D_{f_i}$.
	
	$\bigcup_i D_{f_i} =\spec R$ means $\bigcap_i V(f_i) = V(1)$. Since $\bigcap_i V(f_i)= V(\sum_i (f_i))$, there are elements $r_i \in R$ such that 
	\[
	\sum_i r_i f_i = 1
	\] and only finite $a_i$ are not 0. Hence the correspondent finite $D_{f_k}$ form a finte cover of $\spec R$.
\end{proof}