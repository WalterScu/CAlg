% !TEX root=./CAlg.tex
\section{Topological properties of prime spectrum}
Let $R$ be a ring.
\[
\spec R = \big\{\text{all prime ideals of }R  \big\}
\]
We can give $\spec R$ topology called Zariski topology, which is the origin of modern algebraic geometry. Closed sets of Zariski topology is defined with form 
\[
V(I) = \big\{ \mathfrak{p} \in \spec R | I \subseteq \mathfrak{p}\big\}
\]
where $I$ is arbitrary ideal of $R$.

It is homework to check that these subsets of $\spec R$ satisfy closed set axioms
\begin{enumerate}
\item $V(I_1) \cup V(I_2 )= V(I_1 \cap I_2)$ 
\item $\bigcap V(I) = V(\sum I)$
\item $\emptyset = V(1)$ and $\spec R = V(0)$
\end{enumerate}

\begin{ex}
$\spec \mathbb{Z}$

$\spec k[ x]$, where $k$ is a algebraic closed field.
\end{ex}
\[ D(I) = \spec R \setminus V(I)\]
are used to denote the open sets of $\spec R$ in Zariski topology. In particular, if $I =(f)$ is principal ideal generated by $x$, then $D(I)= D(f)$ is called a principal open set.
\begin{secprop}
$V(I) = V(\sqrt{I}); D(I)= D(\sqrt{I})$ 
\end{secprop}
\begin{proof}
If $I$ is contained in a prime ideal $\mathfrak{p}$, then $ \forall x \in \sqrt{I}, \exists n >0, x^n \in I$. So $x^n \in \mathfrak{p}$. Since $\mathfrak{p}$ is prime, $x \cdot x^{n-1} \in \mathfrak{p}$ implies that $x \in \mathfrak{p}$ or $x^{n-1} \in \mathfrak{p}$, then by induction of $n$, we can conclude that $x \in \mathfrak{p}$. Hence $\sqrt{I} \subseteq \mathfrak{p}$. Hence $V(I) = V( \sqrt{I})$.
\end{proof}
\begin{secprop}
Principal open subsets of $\spec R$ form topology base of Zariski topology.
\end{secprop}
\begin{proof}
\begin{enumerate}
\item $\bigcup_{f \in I} = D(I)$ for any open subset $D(I)$ of  $\spec R$. 
\item For any ideal $J$, there is a principal ideal $(f)$ such that $D_f \subseteq D(I)$. For example, we can choose any element $f$ of $I$.
\end{enumerate}
\end{proof}

\begin{secprop}
$\spec R$ is quasi-compact.
\end{secprop}
\begin{proof}
	It is sufficient to prove for any principal open cover $\bigcup_i D_{f_i}$.
	
	$\bigcup_i D_{f_i} =\spec R$ means $\bigcap_i V(f_i) = V(1)$. Since $\bigcap_i V(f_i)= V(\sum_i (f_i))$, there are elements $r_i \in R$ such that 
	\[
	\sum_i r_i f_i = 1
	\] and only finite $a_i$ are not 0. Hence the correspondent finite $D_{f_k}$ form a finte cover of $\spec R$.
\end{proof}
\begin{secprop}[Separability of $\spec R$]
	$\spec R$ is $T_0$ and $T_1$ as following figures show.
\end{secprop}
\begin{rem}
	$\spec R$ is not Hausdorff if $R \neq 0$. Recall that in a Hausdorff space, each single point is closed point, i.e. $\overline{\text{pt}} = \text{pt}$. $(0) \in D_f, \forall f \in R, f \neq 0$, since $(0) \subsetneq (f)$. Hence $\{(0)\}$ is dense in $\spec R$. Hence $\{(0)\}$ is not closed, it implies that $\spec R$ is not Hausdorff. $(0)$ is called a \textbf{generic point} of $\spec R$.
\end{rem}
\begin{secprop}
	$\mathfrak{p} \in \spec R$ is closed if and only if $\mathfrak{p}$ is maximal ideal.
\end{secprop}
\begin{proof}
	Notice that 
	\[
	\overline{\{\mathfrak{p}\}} = \bigcap_{I \subseteq \mathfrak{p}}V(I) = V(\sum_{I \subseteq \mathfrak{p}}I) = V(\mathfrak{p})
	\]
	Hence $\overline{\{\mathfrak{p}\}}= \{ \mathfrak{p} \}$ implies $\mathfrak{p}$ is the unique ideal which contains $\mathfrak{p}$. Since $\mathfrak{p}$ is contained in a maximal ideal, and maximal ideals are all primes ideals, then $\mathfrak{p}$ is maximal.
\end{proof}
Suppose $A, B$ be two rings. $x,y \in \varphi^{-1} (\mathfrak{p})$ means $\varphi(xy) \in \mathfrak{p} \subideal A$. Hence $\varphi(x) \in \mathfrak{p}$ or $\varphi(y) \in \mathfrak{p}$. Hence $x \in \varphi^{-1} ( \mathfrak{p})$ or $y \in \varphi^{-1}(\mathfrak{p})$. 
By upper statement, we can conclude that
\[
\begin{aligned}
	\varphi^* : \spec B &\rightarrow \spec A&\\
	\varphi^* (\mathfrak{p}) : &= \varphi^{-1}(\mathfrak{p})
\end{aligned}
\]
is well defined.
\begin{secprop}
	\begin{enumerate}
		\item $\varphi^*$ is continuous under Zariski topology.
		\item $\varphi^*$ is dominant if and only if $\ker \varphi \subseteq \text{Rad}(R)=\sqrt{(0)}$
	\end{enumerate}
\end{secprop}
\begin{proof}
	\begin{enumerate}
		\item $(\varphi^*)^{-1}(V(f))= \big\{ \mathfrak{p} \in \spec B | f \in\varphi^*(\mathfrak{p}) = \varphi^{-1}(\mathfrak{p}) \big\}$. It is equal to $V(\varphi(f))= \big\{\mathfrak{p} \in \spec B| \varphi(f) \in \mathfrak{p}  \big\}$. Hence the inverse image of any closed subset is also closed. This means $\varphi^*$is continuous.
		\item Notice that 
		\[
		\varphi^*(\spec B) = \big\{ \varphi^{-1}(\mathfrak{p})| \mathfrak{p} \in \spec B)\big\} \subseteq V(\bigcap_{ \ker \varphi \subseteq \mathfrak{p}} \mathfrak{p}) = V(\sqrt{\ker \varphi}) \subseteq \spec A
		\]
		If $\overline{\varphi^* (\spec B)} = \spec A$. then by upper fact, we know $\sqrt{\ker \varphi}= \text{Rad} A$ since they are all radical ideal. This implies $\ker \varphi \subseteq \text{Rad} A$
		
		Conversely, 
		\[
		\overline{\varphi^* (\spec B)} = \overline{\varphi^* (V(0))} = V(\varphi^{-1}(0) )= V(\ker \varphi).
		\]
		Hence $\ker \varphi \subseteq \text{Rad}A$ implies $\spec A \subseteq V(\ker \varphi)$. Hence $\overline{\varphi^* (\spec B)} = \spec A$.
	\end{enumerate}
\end{proof}
In particular, if $A$ is integral, then $\varphi^* : \spec B \rightarrow \spec A$ is dominant if and only if $\varphi$ is injective.
\begin{ex}
	Let $k$ be a algebraic closed field,
	\[
	\begin{aligned}
	X&=\spec k[x_1, \cdots, x_n]/(f_1)& & \sheaf{X}= k[x_1, \cdots, x_n]/(f_1)\\
	Y&=\spec k[x_1, \cdots, x_n]/(f_2)& &\sheaf{Y}=k[x_1, \cdots, x_n]/(f_2)
	\end{aligned}
	\]
	where $f_1, f_2$ are inreducible polynomials over $k$. Denote $K_X$ and $K_Y$ the fraction field of $\sheaf(X)$ and $\sheaf{Y}$, they are called \textbf{function fields} of $X$ and $Y$. Obviously, $\sheaf{X}$ and $\sheaf{Y}$ are integral, hence by upper propsition, if $\varphi: Y \rightarrow X$ is dominant, then $\varphi$ induce an embedding from $\sheaf{X}$ to $\sheaf{Y}$. Furthermore, there is also an embedding from $K_X$ to $K_Y$, which can be looked as a field extension. It is also used to give a definition of dimesions for reduced irreducible affine schemes
\end{ex}

