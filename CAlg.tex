% !TEX root=CAlg.tex
\documentclass{amsart}
\usepackage{notestemplate}
\usepackage[all]{xy}
\usepackage{tikz-cd}
\title{Seminar Notes: Commutative Algebras}
\author{Zou Haitao}
\begin{document}
\maketitle
\tableofcontents
\section{Rings and Ideals}
\begin{secdefn}
A \textbf{ring} $R$ is a set with two maps (addition) $+ : R \times R \rightarrow R$, (multiplication)$\times : R \times R \rightarrow R$( denote $+(x,y)$ by $x+y$ and $\times (x,y)$ by $x \times y$)that satisfy following properties
\begin{enumerate}
\item R is a abelian group with respect to addition, its identity is denoted by $0$;
\item R is a monoid with identity $1 \in R$ with respect to multiplication;
\item $z \times (x+y)= z \times x + z \times y$ and $(x+y) \times z = x \times z + y \times z$ for any given $x,y,z$.
\end{enumerate}
We typically write $xy$ for $x \times y$.
\end{secdefn}
In a ring $R$, if $1=0$, then $R$ has only one elements, it is trivial and called \textbf{zero ring}. Denoted zero ring by 0.

 Suppose $R$ be a ring. R is commutative if for any $x,y \in R$, $xy=yx$. Rings mentioned in this notes will always be commutative other assumption.
\begin{secdefn}
Let $A$ and $B$ be two rings. $1_A$ and $1_B$ are their identities. A ring homomorphism from $A$ to $B$ is a map $f: A \rightarrow B$, which preserves both addition and multiplication structure, that means, for any $x,y \in A$
\[
\begin{aligned}
f(x +y) &=f(x) +f(y)\\
f(xy) &= f(x) f(y)\\
f(1_A)&=1_B
\end{aligned}
\]
\end{secdefn}
 Suppose $f: A \rightarrow B$ be a ring homomorphism. We have $f(0_A) = f(1_A - 1_A)=f(1_A) - f(1_A)= 1_B - 1_B = 0_B$.
 \begin{enumerate}
 \item If $f$ is surjective as map, then $f$ is called surjective homomorphism.
 \item If $f$ is injective as map, then $f$ is called injective homomorphism.
 \end{enumerate}
\begin{secdefn}
An \textbf{isomorphism} between two rings $A$ and $B$ is a ring homomorphism $f: A \rightarrow B$ such that there is another ring homomorphism $g:B \rightarrow A$ satisfying
\[
\begin{aligned}
f \circ g = \id{B} & & g \circ f = \id{A}
\end{aligned}
\]
\end{secdefn}
\begin{rem}
$f$ is isomorphism if and only if  $f$ is both surjective and injective as ring homomorphism.
\end{rem}
\begin{proof}
If $f$ is isomorphism, then $f(x) = f(y)$ implies $g(f(x))=g(f(y))$. But $g \circ f = \id{A}$, so $x=y$. Hence $f$ is injective. For any $b \in B$, $b= f \circ g(b)$ since $f \circ g= \id{B}$. Let $a= g(b), b=f(a)$. That means $f$ is surjective. 

If $f$ is both injective and surjective homomorphism, then we only need to check if $f^{-1}$ is ring homomorphism. $f(f^{-1}(b_1 + b_2)) = b_1 + b_2 = f \circ f^{-1} (b_1) + f \circ f^{-1}(b_2)= f(f^{-1}(b_1)+ f^{-1}(b_2))$. Since $f$ is surjective, $f^{-1}(b_1 + b_2) = f^{-1}(b_1) + f^{-1}(b_2)$.

Similarly, $f^{-1}(b_1 b_2) = f^{-1}(b_1) f^{-1}(b_2)$.
\end{proof}
It is not always true in arbitrary category(\textbf{Top}, $\textbf{Sch}/k$, $\textbf{Mod}_k$, etc).

If two rings are isomorphic, then we view them as same object in ring category.
\begin{secdefn}
Let $R$ be a ring. We call $i:\tilde{R}  \rightarrow R$ is a subring if $i$ is injective ring homomorphism, written as $\tilde{R} \subset R$
\end{secdefn}
\begin{rem}
	The definition of subring in "Atiyah\& MacDonald" is not exact since it doesn't requrie $\tilde{R}$ to be even a ring.
\end{rem}

\begin{rem}
	$i(\tilde{R}) \simeq \tilde{R}$, so $\tilde{R}$ can be viewed as $i(\tilde{R})$ whose elements are in $R$.
\end{rem}
\begin{secdefn}
	Let $R$ be a ring, $I$ be an additive subgroup of $R$. $I$ is called an \textbf{ideal} of $R$ if for any $r \in R$
\[
\begin{aligned}
&rI := \{ ra| a \in I \} \subset I&\\
&Ir := \{ ar| a \in I \} \subset I&
\end{aligned}
\]
Since $R$ is commutative, $Ir = rI$. We only need to check one of them. If $I$ is ideal of $R$, then we denote the fact by $I \subideal R$.

An ideal $\mathfrak{p} \subideal R$ is called \textbf{prime ideal} if $xy \in \mathfrak{p}$ implies either $x \in \mathfrak{p}$ or $ y \in \mathfrak{p}$.

An ideal $\mathfrak{m} \subideal R$ is called \textbf{maximal ideal} if $\mathfrak{m} \neq (1)$ and if there is no ideal $I$ such that $m \subsetneq I \subsetneq (1)$.
\end{secdefn}

\[
\text{Ideal}(R):= \bigl\{ \text{ideals of } R\bigr\}
\]

Let $\varphi: A \rightarrow B$ be a ring homomorphism. Then there is induced map
\[
\begin{aligned}
\varphi^{\#}: \text{Ideal}(B) &\rightarrow \text{Ideal}(A)\\
\mathfrak{b} &\mapsto \varphi^{-1}(\mathfrak{b})
\end{aligned}
\] 
For any $x,y \in \varphi^{-1}(\mathfrak{b})$, $\varphi(x+y)= \varphi(x) + \varphi(y) \in \mathfrak{b}$, $\varphi(ax)= \varphi(a)\varphi(x) \in \mathfrak{b}$ implies that $ax \in \varphi^{-1}(b)$. Hence $\varphi^{-1}(\mathfrak{b}) \in \text{Ideal}(A)$. Furthermore, it can be checked that $\varphi^{\#}$ is map from $\spec B$ to $\spec A$.

\[
\ker \varphi:= \bigl\{ a \in A| \varphi(a)=0\bigr\}= \varphi^{-1}((0))
\]
If $a_0 \in \ker \varphi$, then for any $a \in A$, $\varphi(a a_0) = \varphi(a) \varphi(a_0) =0$, so $a a_0 \in \ker \varphi$. Hence $\ker \varphi \in \text{Ideal}(A)$.
Since $0$ is contained in any ideals, $\varphi^{\#}(\mathfrak{b})= \varphi^{-1}(\mathfrak{b}) \supset \ker \varphi$

\begin{seclemma}
	Let $I \subideal R$. Relation such that $\sim_{I}$ on R defined as $ x \sim_{I} y$ if and only if $x-y \in I$ is a equivalence relation.
\end{seclemma}
\begin{proof}
	
	\begin{enumerate}
		\item $x-x = 0 \in I \Rightarrow x \sim_{I} x$
		\item $x-y \in I \Rightarrow y-x = -(x-y) \in I \Rightarrow y \sim_{I}x$ 
		\item $x \sim_{I} y, y \sim_{I} z \Rightarrow x-y \in I, y-z \in I \Rightarrow x-z =(x-y) + (y-z) \in I \Rightarrow x \sim_{I} z$.
	\end{enumerate}
\end{proof}

\begin{secdefn}
	Let $I$ be a ring
	\[
	\begin{aligned}
		R/I &:= (R \big/ \sim_{I}, \times, +)\\
		\bar{x} + \bar{y} &= \overline{x+y}\\
		\bar{x} \times \bar{y} &= \overline{xy}
	\end{aligned}
	\]
	is called quotient ring of $R$ by ideal $I$.
\end{secdefn}
\begin{rem}
	It is easy to check $R/I$ is well defined
	\[
	\begin{aligned}
	\varphi: R &\rightarrow R/I\\
	r &\mapsto \bar{r}
	\end{aligned}
	\]
	$\varphi(r_1 + r_2)= \overline{r_1 + r_2} = \bar{r_1} + \bar{r_2}= \varphi(r_1) + \varphi(r_2)$, $\varphi(r_1 r_2) = \overline{r_1 r_2} = \bar{r_1} \bar{r_2} = \varphi(r_1) \varphi(r_2)$ and $\overline{1_R} \bar{r} = \overline{1_R r} = \bar{r}$, so $\varphi(1_R)= \overline{1_R}$ is identity of $R/I$. Hence $\varphi$ is ring homomorphism. 
\end{rem}
FACT: 
\begin{enumerate}
	\item $\ker \varphi = I$;
	\item $\varphi$ is surjective;
	\item $\varphi^{\#}$ is injective. If $\varphi^{\#}(\bar{\alpha}) = \varphi^{\#}(\bar{\beta})$, then $\varphi^{-1}(\bar{\alpha})=\varphi^{-1}(\bar{\beta})$. $\varphi$ is surjective so $\bar{\alpha} = \bar{\beta}$.
	\item If $\ker \varphi \subset I \subideal R$, then for any $\bar{i} \in \varphi(I)$, $\bar{r} \bar{i} = \overline{ri} = \varphi(ri)$ and $\varphi(I)$ is additive subgroup of $R/I$, $\varphi(I) \in \text{Ideal}(R/I)$. $\varphi$ is surjective, so $I= \varphi^{-1}(\varphi(I))= \varphi^{\#}(\varphi(I))$.
\end{enumerate}
(3) and (4) implies following proposition.
\begin{secprop}
	$\varphi^{\#}$ is one-to-one correspondence between $\text{Ideal}(B/I)$ and set of ideals contain $I$ in $R$.
\end{secprop}
\qed
\begin{secdefn}
	Let $R$ be a ring.
	\begin{enumerate}
		\item $x \in R$ is called \textbf{zero divisor} if there is $r \in R, r \neq 0$ such that $rx=0$.
		\item $x \in R$ is called \textbf{nilpotent element} if $x^n=0$ for some $n > 0$.
		\item $x \in R$ is an \textbf{unit} of $R$ if $x$ has inverse under multiplication.
		\item If $R$ has no zero divisors except 0, then $R$ is called \textbf{integral domain}.
	\end{enumerate}	
\end{secdefn}
\begin{rem}
	A nilpotent element in a ring is always zero divisor since $x x^{n-1}=0$. If $x$ is a unit in $R$, then $x$ is not a zero divisor. Conversely, it is not always true.
\end{rem}

\begin{secdefn}
	A \textbf{principal ideal} of $R$ is an ideal that can be generated by one element, written as $(x)$, where $x$ is the generator.
\end{secdefn}
For simple example, $(3,6) \subideal \mathbb{Z}$ is principal ideal generated by 6. $R$ itself is also a principal ideal since it can be generated by 1, written as $(1)$.

Let $I_1 \subideal R, I_2 \subideal R$. We give following serveral constructions of ideals
\[
\begin{aligned}
I_1 \cdot I_2 = \{ xy \in R| x \in I_1, y \in I_2\}& & \prod_{i=1}^{n} I_i &= \{ x_1 x_2 \cdots x_n \in R| x_i \in I_i\}&\\
I_1 + I_2 = \{x+y| x\in I_1, y \in I_2 \}& & \sum_{\alpha}I_\alpha &= \{\sum_{\alpha} x_\alpha | x_\alpha \in I_\alpha \text{ and only finite }x_\alpha \text{ are not zero} \}&
\end{aligned}
\]
$I_1 \cap I_2$ is obviously an ideal since $\forall x, y \in I_1 \cap I_2, r\in R, xr \in I_1 \cap I_2$ and $x+y \in I_1 \cap I_2$.

\begin{ex}
	Let $A= \mathbb{Z}$, $(m),(n)$ two principal ideal generated by $m$ and $n$.\\
	$(m)+(n) = ((m,n))$ is generated by $(m,n)$, the g.c.d of $m$ and $n$\\
	$(m)\cdot(n) = (m\cdot n)$\\
	$(m) \cap (n) =([m,n])$ is generated by $[m,n]$, the l.c.d of $m$ and $n$.\\
	If $(m,n)=1$, then $(m)+(n)=(1), (m)(n)=(m)\cap(n)$.
\end{ex}
Let $I \subideal R , x\in R$. $(x,I)$ is ideal generated by $x$ and elements of $I$. Since $(x)+I$ is minimal ideal contains both $x$ and elements of $I$, $(x)+I =(x,I)$.
\begin{secdefn}
	If $I_1 \subideal R, I_2 \subideal R$. $I_1$ and $I_2$ are called \textbf{coprime} if $I_1 + I_2 = (1)$.
\end{secdefn}

\begin{secprop}
	\label{prop1}
	If $I_1 \subideal R, I_2 \subideal R$ are coprime, then $I_1 \cdot I_2 = I_1 \cap I_2$.
\end{secprop}
\begin{proof}
	By definition, $I_1 \cdot I_2 \subseteq I_1 \cap I_2$. Let $x \in I_1 \cap I_2$, $x$ can be represented by $x=ar_1 +b r_2$, where $a \in I_1, b\in I_2$. Hence $x \in I_1 \cdot I_2= I_1 \cap I_2$.
\end{proof}

\begin{secdefn}
	Let $A_\alpha$ be a family of rings. Their \textbf{direct product} is defined as object $ \prod_{\alpha}A_{\alpha}$ in \textbf{Rings} satisfying following universal property
	\[
	\begin{tikzcd}
	&  & A_i \\
	Z \arrow[rru, "h_i", bend left] \arrow[r, "\exists! " description, dotted] \arrow[rrd, "h_j"', bend right] & \prod_{\alpha} A_{\alpha} \arrow[ru, "p_i"'] \arrow[rd, "p_j"] &  \\
	&  & A_j
	\end{tikzcd}
	\]
	If $\alpha$ is finite, then elements of $\prod_{\alpha}A_{\alpha}$ can be written as $(x_1, \cdots ,x_n)$, $x_i \in A_i$ for some $n$.
	\[
	\begin{aligned}
	(x_1, \cdots, x_n) \cdot (x'_1, \cdots, x'_n)&= (x_1 x'_1, \cdots, x_n x'_n)&\\
	(x_1, \cdots, x_n) +(x'_1, \cdots, x'_n)&= (x_1 + x'_1, \cdots, x_n + x'_n)&\\
	1&=(1_{A_1}, \cdots, 1_{A_n})&
	\end{aligned}
	\]
\end{secdefn}

Let $A$ be a ring and $\ff{a}_1, \cdots, \ff{a}_n$ ideals of $A$. Define a homomorphism
\[
\phi: A \rightarrow \prod_{i=1}^{n}(A/\ff{a}_i)
\]
by rules $\phi(x)= (x+\ff{a}_1, \cdots, x+\ff{a}_n)$.

\begin{secprop}[\ref{}]
	\begin{enumerate}
		\item If $\ff{a}_i, \ff{a}_j$ are coprime whenever $i \neq j$, then $\prod_{i} \ff{a}_i = \bigcap_{i} \ff{a}_i$;
		\item $\phi$ is surjective $\Leftrightarrow$ $\ff{a}_i,\ff{a}_j$ are coprime whenever $i \neq j$;
		\item $\phi$ is injective $\Leftrightarrow$ $\bigcap_{i} \ff{a}_i =(0)$.
	\end{enumerate}
\end{secprop}

\begin{proof}
	\begin{enumerate}
		\item By \ref{prop1} the case $n=2$ is proved. Assume it is true when $n=k$. When $n= k+1$, since $\ff{a}_i$ and $\ff{a}_{k+1}$ are coprime for $1 \leq i \leq k$, $\ff{a}_i + \ff{a}_{k+1}= (1)$. It implies that $x_i + y_i =1$ for some $x_i \in \ff{a}_i$, $y_i \in \ff{a}_{k+1}, 1 \leq i \leq k$
		\[
			\prod_{i=1}^{k} x_i =1 \text{ in } A/\ff{a}_{k+1}
		\]
		that means $\prod_{i=1}^{k}x_i + x_{k+1}=1$ for some $x_n \in \ff{a}_{k+1}$ in $R$. Hence $\prod_{i=1}^{k} \ff{a}_i$ and $\ff{a}_{k+1}$ are coprime. Then 
		\[
		\prod_{i=1}^{k+1}\ff{a}_i = ( \prod_{i=1}^{k} \ff{a}_i) \cdot \ff{a}_{k}= (\bigcap_{i=1}^{k} \ff{a}_i) \cap \ff{a}_{k+1} = \bigcap_{i=1}^{k+1} \ff{a}_i
		\]
		by induction.
		\item If $\phi$ is surjective, then there exists $x \in A$ such that $\phi(x)= (\delta_{1}^{i}, \cdots, \delta_{n}^{i})$.
		 Hence $x \equiv 1 \mod \ff{a}_i, x \equiv  0 \mod \ff{a}_j$ whenever $i \neq j$. So
		 \[
		 (1-x)+ x = 1
		 \]
		 where $1-x \in \ff{a}_i, x \in \ff{a}_j$. Hence $\ff{a}_i$ and $\ff{a}_j$ are coprime.
		 
		 Since $\prod_{i=1}^{n}(A/\ff{a}_i)$ can be linear represented by $(\delta_j^i)_{j=1}^n, 1 \leq i \leq n$, it is enough to show for any $(\delta_j^i)_{j=1}^n$, there is $x_i \in R$ such that $\phi(x_i)=(\delta_{j}^{i})_{j=1}^n$.
		 
		 Since $\ff{a}_i$ and $\ff{a}_j$ are coprime for all $j \neq i$, there are equations $x_j + x_i =1, x_j \in \ff{a}_j, y_j \in \ff{a}_i$
		 \[
		 \begin{aligned}
		 \prod_{j \neq i} x_j \equiv 0 &\mod \ff{a}_i&\\
		 \prod_{j \neq i} x_j = \prod_{j \neq i}(1- y_j) \equiv 1 &\mod \ff{a}_j
		 \end{aligned}		 
		 \]
		 whenever $i \neq j$. Hence $\phi(\prod_{j \neq i} x_j) = (\delta_{j}^i)_{j=1}^n$.
		\item $\phi(x)=0$ means that $x \in \ff{a}_i$ for all $1 \leq i \leq n$. Hence it is equivalent to $x \in \bigcap_{i=1}^n \ff{a}_i$. Hence $\phi$ is injective $\Leftrightarrow$ $\ker \phi = (0) \Leftrightarrow \bigcap_{i=1}^{n} \ff{a}_i =0$.
	\end{enumerate}
\end{proof}

Following are equivalent criteria for primes ideals and maximal ideals
\begin{secprop}
	Let $R$ be a ring.
	\begin{enumerate}
		\item $\mathfrak{p} \subideal R$ is prime ideal if and only if $R/\mathfrak{p}$ is integral domain.
		\item $\mathfrak{m} \subideal R$ is maximal ideal if and only if $R/\ff{m}$ is a field.
	\end{enumerate}
\end{secprop}
\begin{proof}
	\begin{enumerate}
		\item Let $\mathfrak{p} \subideal R$ be a prime ideal. For any $x,y \in R$, $\bar{x},\bar{y} = \bar{0}$ is equivalent to $xy \in \mathfrak{p}$. But $xy \in \mathfrak{p}$ implies that either $x \in \mathfrak{p}$ or $y \in \mathfrak{p}$, equivalently, $\bar{x}=0$ or $\bar{y}=0$. This shows that $R/\mathfrak{p}$ is integral domain. 
		
		Conversely, if $R/\mathfrak{p}$ is integral domain, then for any $x,y \in R$ such that $xy \in \mathfrak{p}$, $\bar{x} \bar{y} = \bar{0}$ in $R/\mathfrak{p}$, we have $\bar{x}=\bar{0}$ or $\bar{y} = \bar{0}$. That means $x \in \mathfrak{p}$ or $y \in \mathfrak{p}$. Hence we can conclude the equivalence.
		\item Let $\mathfrak{m} \subideal R$ be a maximal ideal. If $ \bar{x} \in R/\mathfrak{m}, \bar{x} \neq \bar{0}$, then $x \notin \mathfrak{m}$. Since $\mathfrak{m}$ is maximal, $m \subsetneq (\mathfrak{m}, x) \subset (1)$ implies that $(\mathfrak{m},x) = (1)$. That means, there exists $y \in R$ such that $xy +m=1 $ for some $m \in \mathfrak{m}$. Obviously, $y \notin m$, so $\bar{x} \bar{y} = \bar{1}$ in $R/\mathfrak{m}$. Hence each non-zero element in $R/\mathfrak{m}$ is unit. Hence $R/\mathfrak{m}$ is a field.
		
		Conversely, if $R/\mathfrak{m}$ is a field, then $\bar{x} \in R/\mathfrak{m}, \bar{x} \neq 0$ is unit. But $\bar{x} \neq \bar{0}$ is equivalent to $x \notin \mathfrak{m}$ and $\bar{x}$ is unit $R/\mathfrak{m}$ if and only if $x$ is unit in $R$. So $(\mathfrak{m}, x) = (1)$ if $x \notin \mathfrak{m}$. Hence $\mathfrak{m}$ is maximal. The proof is complete. 
	\end{enumerate}
\end{proof}

\begin{secthm}[Krull's theorem]
	If $R$ is a ring and $R \neq 0$, then $R$ has at least one maximal ideal.
\end{secthm}
\begin{proof}
	Since $R \neq 0$, $(0) \in \text{Ideal}(R)$ and $(0) \neq (1)$. We can order $\sum = \text{Ideal}(R)-\{ (1)\}$ by inclusion($I_1 \leq I_2$ iff $I_1 \subseteq I_2$). Suppose $\{ I_\alpha\}$  be a chain in $\sum$,i.e. $\forall I_\alpha , I_\beta \in \{ I_\alpha\}$, $I_ \alpha \leq I_\beta$ or $I_\beta \leq I_\alpha$. Denote $\bigcup_\alpha I_\alpha$ by $I$. $I$ is obviously an ideal and $1 \notin I$ since for each $\alpha, 1 \notin I_\alpha$. Hence $I \in \sum$ and $I$ is upper bound of $\{ I_\alpha \}$. By Zorn's lemma, $\sum$ has at least one maximal element, it is an maximal ideal in $R$ by definition.
\end{proof}

\begin{seccor}
	Let $R$ be ring. If $I \subideal R$ and $I \neq (1)$, then $I$ is contained in one maximal ideal.
\end{seccor}
\begin{proof}
	$R/I \neq 0$. By Krull's theorem, $R/I$ has at least one maximal ideal $\bar{\mathfrak{m}}$. Then $\varphi^{\#}(\bar{\mathfrak{m}})$ is maximal ideal which contain $I$ since $\varphi^{\#}$ induce one-to-one correspondence between $\sum_{R/I}$ and set of non-trivial ideals which contain $I$ and $\varphi^{\#}$ preserves order.
\end{proof}

\begin{seccor}
	Any non-unit in $R$ is contained in a maximal ideal.
\end{seccor}

\begin{secdefn}
	A ring with only one maximal ideal $\mathfrak{m}$ is called a local ring with maximal ideal $\mathfrak{m}$. Suppose $(R, \mathfrak{m})$ be a local ring with maximal ideal $\mathfrak{m}$. $R/\mathfrak{m}$ is called residue field of $R$.
\end{secdefn}

\begin{secprop}[\ref{}]
	\begin{enumerate}[(i)]
		\item Let $A$ be a ring and $\mathfrak{m} \neq (1)$ and ideal of $A$ such that every $x \in A - \mathfrak{m}$ is a unit in $A$. Then $A$ is local ring and $\mathfrak{m}$ its maximal ideal.
		\item Let $A$ be a ring and $\mathfrak{m}$ a maximal ideal of $A$, such that every element of $1+ \mathfrak{m}$ is a unit. Then $A$ is a local ring.
	\end{enumerate}
\end{secprop}
\begin{proof}
	\begin{enumerate}[(i)]
		\item Since elements in $A-\mathfrak{m}$ are all units and every ideal not equal to $(1)$ contains non-unit, all maximal ideals are contained in $\mathfrak{m}$. Hence $\mathfrak{m}$ is maximal ideal and the only one.
		\item Let $x \in A- \mathfrak{m}$. Since $\mathfrak{m}$ is maximal, $(\mathfrak{m},x)=(1)$. That means there exist $y \in A$ and $ m \in \mathfrak{m}$ such that $xy +m =1$. Hence $xy=1-m$ is unit by hypothesis so is $x$. Hence $A$ is local ring by (i).
	\end{enumerate}
\end{proof}

\begin{secdefn}
	Let $R$ be a ring.
	\[
	\mathbf{Rad}(R)=\big\{ r \in R | r \text{ is nilpotent}\big\}
	\]
	is called nilradical of $R$ or simply radical of $R$.
	\[
	\mathbf{JRad}(R)=\big\{ r \in R| \forall y \in R, 1- ry\text{ is unit}\big\}
	\]
	is called Jacobson radical of $R$.
\end{secdefn}
\begin{secprop}
	$\mathbf{Rad}(R)$ is intersection of all prime ideals of $R$; $\mathbf{JRad}(R)$ is intersection of all maximal ideals of $R$.
\end{secprop}

\begin{proof}
	If $\mathfrak{p} \subideal R$ is prime, then $R/ \mathfrak{p}$ is integral. Hence every $x \notin \mathfrak{p}$ is not nilpotent otherwise $\bar{x}$ in $R/\ff{p}$ is also nilpotent. Hence \[\mathbf{Rad}(R) \subseteq \bigcap_{\ff{p} \subideal R\text{ is prime}} \ff{p}\]
	If $x$ is not nilpotent, then we need to prove that there is prime ideal does not contian $x$. Let $S=\{ 1, x, x^2, \cdots\}$ and $\sum$ be the set of ideals that disjoint with $S$. Since $(0) \in \sum $ and $\sum$ is ordered by inclusion, by Zorn's lemma $\sum$ has maximal element, denote it by $\ff{p}$. We need to prove $\ff{p}$ is a prime ideal.
	
	Let $a \notin \ff{p}, b \notin \ff{p}$. $(a, \ff{p})$ and $(b, \ff{p})$ are not elements in $\sum$ since $\ff{p}$ is maximal. That means there exist $m,n \geq 0$ such that $x^m \in (a, \ff{p}), x^n \in (b, \ff{p})$. It implies 
	\[
	\begin{aligned}
	x^m = r_1 a + p_1, x^n = r_2 b +p_2& & r_1, r_2 \in R
	\end{aligned}
	\]
	Hence $x^{m+n} = r_1 r_2 ab + (r_2 bp_1 + r_1 ap_1 + p_1 p_2) \in (ab, \ff{p})$. Hence $(ab, \ff{p}) \in \sum$ and therefore $ab \in \ff{p}$. Hence $\ff{p}$ is prime ideal and $x \in \ff{p}$.
	
	Let $x \in R$. If there is $y\in R$ such that $1-xy$ is not unit, then there is a maximal $\ff{m}$. Hence 
	\[
	x \notin \bigcap_{\ff{m} \subideal R\text{ is maximal}} \ff{m}
	\]
	If $x \in \ff{m}$ for some maximal ideal $\ff{m}$, then $(x, \ff{m}) = (1)$. That means $1=rx + m$ for some $r \in R, m \in \ff{m}$. Hence $1-rx \in \ff{m}$ is not unit. Hence $x \notin \textbf{JRad}(R)$.
\end{proof}
Here we will introduce some essential facts about prime ideals that used frequently in algebraic geometry.

\begin{secprop}[\ref{}]
	\begin{enumerate}[(i)]
		\item Let $\ff{p}_1, \cdots, \ff{p}_n$ be prime ideals and $\alpha$ be ideals contained in $\bigcup_{i=1}^{n} \ff{p}_i$. Then $\ff{a} \subseteq \ff{p}_i$ for some $i$.
		\item Let $\ff{a}_1, \cdots, \ff{a}_n$ be ideals and let $\ff{p}$ be a prime ideal  which contains $\bigcap_{i=1}^{n} \alpha_i$. Then $\ff{a}_i \subseteq \ff{p}$ for some $i$. In particular, if $\ff{p}= \bigcap_{i=1}^n \ff{a}_i$, then $\ff{p} = \ff{a}_i$ for some $i$.
	\end{enumerate}
\end{secprop}

\begin{proof}
	\begin{enumerate}[(i)]
		\item When $n=1$, it is true obviously.
		
		If it is true that $\ff{a} \nsubseteq \ff{p}_i(1 \leq i \leq n)$ for some $n > 0$ can implies $\ff{a} \nsubseteq \bigcup_{i=1}^{n} \ff{p}_i$, then for given $\ff{p}_1, \ff{p}_2, \cdots, \ff{p}_{n+1}$, if $\ff{a} \nsubseteq \ff{p}_i (1\leq i \leq n+1)$, then $\ff{a} \nsubseteq \bigcup_{i \neq j} \ff{p}_i$. So for each $1\leq j \leq n$, there is $x_j \in \ff{a}$ such that $x_j \notin \ff{p}_i$ whenever $i \neq j$.
		
		Let
		\[
		y = \sum_{j=1}^n x_1 x_2 \cdots \hat{x_j} \cdots x_n
		\]
		Since $\ff{p}_j$ is prime, $x_1 x_2 \cdots \hat{x_j} \cdots x_n \notin \ff{p}_j$. Hence $y \notin \ff{p}_j$ for all $1 \leq j \leq n$. But $y \in \ff{a}$, hence $\ff{a} \nsubseteq \bigcup_{j=1}^n \ff{p}_j$. By induction, it is true for all $n >0$.
		
		\item If $\ff{a}_i \nsubseteq \ff{p}$ for all $i$, then there are $x_i \in \ff{a}_i$ such that $x_i \notin \ff{p}$ for all $i$. Since $\ff{p}$ is prime, $x_1,x_2, \cdots, x_n \notin \ff{p}$. But $x_1 x_2 \cdots x_n \in \prod_{i=1}^{n} \ff{a}_i \subseteq \bigcap_{i=1}^{n} \ff{a}_i$. Hence $\bigcap_{i=1}^n \ff{a}_i \nsubseteq \ff{p}$. Hence $\bigcap_{i=1}^n \ff{a}_i \subseteq \ff{p} \Rightarrow \ff{a}_i \subseteq \ff{p}$ for some $i$.
		
		If $\bigcap_{i=1}^{n} \ff{a}_i = \ff{p}$, then $\ff{p} \subseteq \ff{a}_j$ for all $j$. But $\ff{a}_j \subseteq \ff{p}$, so $\ff{a}_i = \ff{p}$.
	\end{enumerate}
\end{proof}

Let $I_1 \subideal R, I_2 \subideal R$.
\[
(I_1: I_2) := \big\{ r \in R| rI_2 \subseteq I_1\}
\]
\begin{ex}
	\begin{enumerate}[1)]
		\item Let $I \subideal R$. $(0: I)=((0):I)=\ann(I)$ is called \textbf{annihilator} of $I$.
		
		If $I=(x)$ is principal ideal, then $\ann((x))$ is shortly denoted by $\ann(x)$, called \textbf{annihilator of $x$}. If $x$ is non-zerodivisor, then $\ann(x) =0$.
		\item Let $R= \mathbb{Z}$, $I_1=(m), I_2=(n)$. $m= \prod_{i=1}^{n} p_i^{\alpha_i}, n = \prod_{i=1}^{n} p_i^{\beta_i}$ are prime decomposition by $p_1, \cdots, p_n$. Let $\gamma_i = \max \{ \alpha_i -\beta_i, 0\}$, then $(I_1: I_2)= ( \prod_{i=1}^{n} p_i^{\gamma_i})$ and $\prod_{i=1}^{n} p_i^{\gamma_i} = \frac{m}{(m,n)}$.
	\end{enumerate}
\end{ex}

\begin{secdefn}
	Let $I \subideal R$.
	\[
	\sqrt{I} = \big\{x \in R| x^n \in I \text{ for some } n \big\}
	\]
	is called radical ideal of $I$.
\end{secdefn}
\begin{rem}
	$\textbf{Rad}(R)= \sqrt{(0)}$
\end{rem}

\begin{secprop}
	$\sqrt{I}$ is the intersection of all primes ideals which contain $I$.
\end{secprop}
\begin{proof}
	\[
	\begin{aligned}
		 & x \in \sqrt{I}&\\
		 \Leftrightarrow& \exists n >0 , x^n \in I&\\
		 \Leftrightarrow& \exists n >0, (\bar{x})^{n} \textbf{ in } R/I&\\
		 \Leftrightarrow& \bar{x} \in \textbf{Rad}(R/I) \text{ is intersection of prime ideals in } R/I&\\
		 \Leftrightarrow& x= \varphi^{-1}(\bar{x}) \in \varphi^{\#}(\textbf{Rad}(R/I))\text{ is intersection of prime ideals which contain }I \text{ in } R&\\
	\end{aligned}
	\]
\end{proof}

\begin{secprop}
	$D=$ set of zero-divisors of $A= \bigcup_{x \neq 0} \sqrt{(\ann(x))}$  
\end{secprop}
\begin{proof}
	First, $\sqrt{D} = D$ since $D$ is prime.
	
	Next, $\sqrt{\cup_{\alpha} E_\alpha} = \cup_{\alpha} \sqrt{E_\alpha}$ for any family of subset of $R$
	Hence $D= \sqrt{D} = \sqrt{\bigcup_{x \neq 0} \ann(x)} = \bigcup_{x \neq 0} \sqrt{\ann(x)}$.
\end{proof}
\end{document}